\documentclass[handout]{beamer}

\title{Four Features You May Not Realize Are Part of Django}
\author{David Fischer}
\date{May 25, 2017}

%% Beamer Themes
\usetheme{Berlin}
\usecolortheme{dove}
\usefonttheme{serif}

%% Packages
% Pygments must be accessible to use minted and --shell-escape
%  must be used with pdflatex
\usepackage{minted}
\usepackage{hyperref}
\usepackage[font=scriptsize,labelformat=empty]{caption}


\setbeamertemplate{footline}{
  \hspace*{.2cm}
  \scriptsize{
    \insertshorttitle
    \hspace*{50pt}
    \hfill
    \insertframenumber/\inserttotalframenumber
    \hspace*{.2cm}
  }
  \vspace{9pt}
}


\begin{document}

\maketitle


% Password validation - new in Django 1.9
% - ties into the user forms (UserCreationForm, PasswordChangeForm, etc.)
% - there are builtin validators for common passwords, length, numeric only passwords and similarity with the username
% - you can add your own!


% Fragile is required for syntax highlighting
\begin{frame}[fragile]
\frametitle{Password validation}

{\tiny
\begin{minted}{python}
## settings.py

# Password validation
# https://docs.djangoproject.com/en/1.11/topics/auth/passwords/#module-django.contrib.auth.password_validation
AUTH_PASSWORD_VALIDATORS = [
    {
        'NAME': 'django.contrib.auth.password_validation.MinimumLengthValidator',
        'OPTIONS': {
            'min_length': 8,
        }
    },
    {
        'NAME': 'django.contrib.auth.password_validation.CommonPasswordValidator',
    },
]
\end{minted}
}

\end{frame}


% IP Geolocation - revised in Django 1.9

% Fragile is required for syntax highlighting
\begin{frame}[fragile]
\frametitle{IP Geolocation}

{\tiny
\begin{minted}{python}

# Useful in views/middleware/utils
# https://docs.djangoproject.com/en/1.11/ref/contrib/gis/geoip2/#django.contrib.gis.geoip2.GeoIP2
from django.contrib.gis.geoip2 import GeoIP2

g = GeoIP2()
g.city('72.14.207.99')

# {'city': 'Mountain View',
#  'country_code': 'US',
#  'country_name': 'United States',
#  'dma_code': 807,
#  'latitude': 37.419200897216797,
#  'longitude': -122.05740356445312,
#  'postal_code': '94043',
#  'region': 'CA',
# }


# * Uses MaxMind GeoIP datasets

\end{minted}
}

\end{frame}


% UUIDs - new in Django 1.8
% These have advantages like making your primary keys not guessable
% They will be unique across database servers in case you ever need to shard data
% Postgres has a native field for this

% Fragile is required for syntax highlighting
\begin{frame}[fragile]
\frametitle{Universal Unique Identifiers as Primary Keys}

{\tiny
\begin{minted}{python}
## models.py

# https://docs.djangoproject.com/en/1.11/ref/models/fields/#django.db.models.UUIDField
class BaseModel(models.Model):
    """
    An abstract base model that defines our PK as a UUID and adds a
    `created_date` and `updated_date` to each model.
    """
    id = models.UUIDField(primary_key=True, default=uuid.uuid4, editable=False)
    created_date = models.DateTimeField(auto_now_add=True)
    updated_date = models.DateTimeField(auto_now=True)

    class Meta:
        abstract = True
\end{minted}
}

\end{frame}


% Class-based Authentication Views - new in Django 1.11
% Allows fine-tuning of Django's login/logout/password change structure

% Fragile is required for syntax highlighting
\begin{frame}[fragile]
\frametitle{Class-based Authentication Views}

{\tiny
\begin{minted}{python}
## views.py

from django.contrib.auth import views as auth_views

# https://docs.djangoproject.com/en/1.11/topics/auth/default/#all-authentication-views
class LoginView(auth_views.LoginView):
    template_name = 'authentication/login.html'
    # Redirect already logged in users to `settings.LOGIN_REDIRECT_URL`
    redirect_authenticated_user = True

    def form_valid(self, form):
        response = super().form_valid(form)
        # User is now logged in - do something extra like send emails, analytics, etc.
        return response
\end{minted}
}

\end{frame}


% I will put a link to this talk in the comments of the meetup

\begin{frame}
\frametitle{}
  {\huge Comments or Questions?}
\end{frame}


\end{document}
